\documentclass[12pt]{article}
\usepackage[english]{babel}
% \usepackage[utf8x]{inputenc}
\usepackage[T1]{fontenc}
\usepackage{scribe}
\usepackage{listings}
\usepackage{fullpage}
\usepackage{amsfonts}
\usepackage{amssymb}

\usepackage{hyperref}
\hypersetup{
    colorlinks=true,
    linkcolor=blue,
    filecolor=magenta,      
    urlcolor=cyan,
    pdftitle={Overleaf Example},
    pdfpagemode=FullScreen,
    }
\usepackage{xurl}
\usepackage[svgnames]{xcolor}
\usepackage{color}
% \definecolor{light-gray}{gray}{0.90}
% \lstset{backgroundcolor=\color{light-gray},showlines=true}

% \usepackage{xcolor}
% \usepackage{listings}

% \lstdefinestyle{BashInputStyle}{
%   language=bash,
%   basicstyle=\small\sffamily,
%   numbers=left,
%   numberstyle=\tiny,
%   numbersep=3pt,
%   frame=tb,
%   columns=fullflexible,
%   backgroundcolor=\color{yellow!20},
%   linewidth=0.9\linewidth,
%   xleftmargin=0.1\linewidth
% }


\usepackage{minted}
\setminted{fontsize=\footnotesize,baselinestretch=0.5}



\Scribe{}
\Lecturer{Queenie Qiu, John Raiti. Student: Fill your name}
\LectureNumber{8}
\LectureDate{DATE: March 9th. 2023}
\LectureTitle{Pick and Place tasks using physical robot arm}

\lstset{style=mystyle}

\begin{document}
	\MakeScribeTop

%#############################################################
%#############################################################
%#############################################################
%#############################################################

\section{Learning Objectives}
\begin{enumerate}
    \item Learn about waypoint generation in robotic arm trajectories and motion planning.
    
    \item Implement code to perform pick and place tasks in a recycle-sorting scenario.
    
    \item Implement above using the physical robot arm. 
    
\end{enumerate}


\section{Pick and Place task with obstacle avoidance}
\textbf{Deliverables:}
In this lab, you will be required to showcase your ability to perform tasks using the robot arm through coding. If you find it too difficult, you can opt to use the Web App or RViz to control the motion instead. However, you need to mention this in your deliverable. You have two task options to choose from.
\begin{enumerate}
    
    \item The first option is to submit a video showcasing your completion of the pick and place task with obstacles using the physical robot arm, along with a written paragraph describing your demonstration. To prepare for this option, it is recommended to complete section 3 of the Pick and Place task with obstacle avoidance in lab7, to become familiar with the procedures of pick and place tasks. We will provide you with a workspace for this task.
    
    Solution:
    Your video link is
    
    Your description is:
    
    \item You have the option to create a customized video using the physical robot arm. While you can choose any task to demonstrate your understanding, it is mandatory to submit a video. You should also include a written paragraph describing your demonstration. 
    
    Solution:
    Your video link is
    
    Your description is:
    
\end{enumerate}

\end{document}